\chapter{Test}
Overordnet set kan man inddele tests i henholdsvis black box test og white box test.\\\\
Black box test, også kendt som funktionel test, handler om, hvorvidt systemet fra et eksternt perspektiv opfører sig som forventet.
Det smarte ved denne test er, at man kender input og output og alt det der sker i systemet er sagen uvedkommende for testeren, og derfor kræver det ikke indblik i koden for at kunne black box teste.\\\\
White box test, også kendt som strukturel test, handler derimod om, hvorvidt systemet fra et indre perspektiv opfører sig som forventet.
Man vælger i white box test at have fokus på programmets logik og dette kan være med til at skabe en høj kodedækning, da man eksempelvis tester, om alle forgreninger kan udføres.
\\Til vores program har vi valgt at lave tre typer tests; unit test, kodedækningstest og monkey test.
Det skal hertil nævnes, at der i projektet er blevet udført integrationstest, da vi ikke bare har valgt en \textit{big bang-approach} og først testet systemet til sidst.
\section{Unit test}
Unit test går i sin helhed ud på at teste de enkelte metoder i spillet, det er altså derfor en white box test. \\
Vores unit tests er dokumenteret i koden, således man kan se opbygningen på hvert enkelte test.
Nedenstående uddrag af vores kode beskriver en unit test af getAmount() og startAmount():\\
\begin{lstlisting}
    @Test
    public void getAmount() throws Exception {
        Stash stash1 = new Stash();
        Stash stash2 = new Stash(100);

        Assert.assertEquals(0, stash1.getAmount());
        Assert.assertEquals(100, stash2.getAmount());
    }
\end{lstlisting}
\\
Først oprettes stash1 af klassen Stash, og da den ikke initaliseres med en værdi, så forventer vi, at getAmount() returnerer en værdi på 0.
Ligeledes når stash2 bliver oprettet, så bliver den intialiseret med 100, og derfor forventes der også en returværdi på 100, når getAmount() kaldes.

\section{Coverage test}
Spillet er blevet kodedækket på to forskellige måder.
Den første måde gik ud på, at vi kørte det normale spil med fire spillere, og vi fik kodedækningsprocent på: Klasser brugt = 100\%, metoder brugt = 78\%, linier brugt = 79\%.
Vi var imidlertidigt ikke tilfredse med det her resultat, og vi kunne se, at den lave kodedækning til dels hang sammen med, at man simpelthen ikke nåede at bruge alle metoder.
\\ Vi valgte derfor at ændre bankerot-værdien således, at man skulle have en beholdning på -500, før man tabte. Dette gjorde, at spillet kørte i langt længere tid, og resultatet endte med at blive: Klasser brugt = 100\%, metoder brugt = 85\%, linier brugt = 82\%.
Dette resultat godtagede vi, fordi det viste sig, at mange af metoderne hang sammen med en ikke-implementeret metode, der hang sammen med typekortene, som i dette projekt er blevet udkommenteret.

\section{Monkey test}
Monkey test (dansk: abetesten) er en black box test, hvor man sådan set bare prøver at taste en helt masse og 'stresse' spillet.
Det er en smart metode til at finde fejl ved programmet, og vi har hyppigt brugt den, fundet fejl og undersøgt dem via debug-funktionen.

\section{Testcases}
Vi har udarbejdet en række test cases ud fra kravende som er specificeret i kapitel \ref{sec:requirements}.
Disse test cases bekræfter at kravende er overholdt og programmet kan derfor godkendes bl.a. ud fra disse tests.

\begin{table}[H]
    \begin{center}
        \begin{tabular}{|l|p{8cm}|}
            \hline
            \textbf{Unique ID} & TC1 \\
            \hline
            \textbf{Summary} & Kan programmet modtage de korrekte navne? \\
            \hline
            \textbf{Requirements} & \\
            \hline
            \textbf{Preconditions} & \\
            \hline
            \textbf{Postconditions} & \\
            \hline
            \textbf{Procedure} & \begin{enumerate}
                \setlength\itemsep{0ex}
                \item Start spillet
                \item Indtast navnet "Adam" og tryk ok
                \item Indtast navnet "Eva" og tryk ok
            \end{enumerate} \\
            \hline
            \textbf{Test data} & \begin{itemize}
                \setlength\itemsep{0ex}
                \item Navn 1: Adam
                \item Navn 2: Eva
            \end{itemize} \\
            \hline
            \textbf{Expected result} & Spillerne er blevet oprettet med navn og vises på brættet \\
            \hline
            \textbf{Actual result} & Som forventet \\
            \hline
            \textbf{Status} & Godkendt \\
            \hline
            \textbf{Tested by} & Oliver Storm Køppen \\
            \hline
            \textbf{Date} & 1/12-2017 \\
            \hline
            \textbf{Environment} & IntelliJ on Win10 \\
            \hline
        \end{tabular}
    \end{center}
\end{table}

\begin{table}[H]
    \begin{center}
        \begin{tabular}{|l|p{8cm}|}
            \hline
            \textbf{Unique ID} & TC2 \\
            \hline
            \textbf{Summary} & Afsluttes spillet, når en spiller har en beholdning på <0? \\
            \hline
            \textbf{Requirements} & \\
            \hline
            \textbf{Preconditions} & To spillere er blevet oprettet og en af spillernes beholdning er på <0\\
            \hline
            \textbf{Postconditions} & Spillet afsluttes\\
            \hline
            \textbf{Procedure} & \begin{enumerate}
                \setlength\itemsep{0ex}
                \item Start spillet
                \item Spillerne er oprettet
                \item Spillet køres, indtil den ene spiller har en beholdning på <0
            \end{enumerate} \\
            \hline
            \textbf{Test data} & \begin{itemize}
                \setlength\itemsep{0ex}
                \item Navn 1: "Adam"
                \item Navn 2: "Eva"
            \end{itemize} \\
            \hline
            \textbf{Expected result} & Adams er gået fallit. \\
            \hline
            \textbf{Actual result} & Som forventet \\
            \hline
            \textbf{Status} & Godkendt \\
            \hline
            \textbf{Tested by} & Oliver Storm Køppen \\
            \hline
            \textbf{Date} & 1/12-2017 \\
            \hline
            \textbf{Environment} & IntelliJ on Win10 \\
            \hline
        \end{tabular}
    \end{center}
\end{table}
\begin{table}[H]
    \begin{center}
        \begin{tabular}{|l|p{8cm}|}
            \hline
            \textbf{Unique ID} & TC3 \\
            \hline
            \textbf{Summary} & Ryger man i fængsel, hvis man har et fængselskort? \\
            \hline
            \textbf{Requirements} & Spiller er oprettet, og spilleren har et fængelskort\\
            \hline
            \textbf{Preconditions} & Spilleren er landet på fængselsfeltet\\
            \hline
            \textbf{Postconditions} & Spilleren kan spille videre uden at komme i fængsel\\
            \hline
            \textbf{Procedure} & \begin{enumerate}
                \setlength\itemsep{0ex}
                \item Start spillet
                \item Indtast navnet "Adam" og tryk ok
                \item Adam har modtaget et fængselskort
                \item Adam lander på fængselsfeltet
            \end{enumerate} \\
            \hline
            \textbf{Test data} & \begin{itemize}
                \setlength\itemsep{0ex}
            \end{itemize} \\
            \hline
            \textbf{Expected result} & Spilleren fortsætter spillet uden at havne i fængsel. \\
            \hline
            \textbf{Actual result} & Som forventet \\
            \hline
            \textbf{Status} & Godkendt \\
            \hline
            \textbf{Tested by} & Niklaes Jacobsen \\
            \hline
            \textbf{Date} & 30/11-2017 \\
            \hline
            \textbf{Environment} & IntelliJ on MacOS \\
            \hline
        \end{tabular}
    \end{center}
\end{table}

\begin{table}[H]
    \begin{center}
        \begin{tabular}{|l|p{8cm}|}
            \hline
            \textbf{Unique ID} & TC4 \\
            \hline
            \textbf{Summary} & Vil spillerens pengeholdning ændrer sig i korrespondance til feltets værdi? \\
            \hline
            \textbf{Requirements} & \\
            \hline
            \textbf{Preconditions} & TC1 og spilleren landet på et felt med værdi, som endnu ikke er ejet \\
            \hline
            \textbf{Postconditions} & Spillerens pengebeholdning er nu blevet mindre i relation til køb af felt\\
            \hline
            \textbf{Procedure} & \begin{enumerate}
                \setlength\itemsep{0ex}
                \item Start spillet
                \item Spillere oprettes
                \item Spilleren lander på Isbutikken
            \end{enumerate} \\
            \hline
            \textbf{Test data} & \begin{itemize}
                \setlength\itemsep{0ex}
                \item Spillers startbeholdning er 20
                \item Isbutikken: M2
            \end{itemize} \\
            \hline
            \textbf{Expected result} & Spillernes pengebeholdning er nu 18\\
            \hline
            \textbf{Actual result} & Som forventet \\
            \hline
            \textbf{Status} & Godkendt \\
            \hline
            \textbf{Tested by} & Oliver Storm Køppen \\
            \hline
            \textbf{Date} & 1/12-2017 \\
            \hline
            \textbf{Environment} & IntelliJ on Win10 \\
            \hline
        \end{tabular}
    \end{center}
\end{table}
