\chapter{Test}
Overordnet set kan man inddele tests i henholdsvis black box test og white box test.\\\\
Black box test, også kendt som funktionel test, handler om, hvorvidt systemet fra et eksternt perspektiv opfører sig som forventet.
Det smarte ved denne test er, at man kender input og output og alt det der sker i systemet er sagen uvedkommende for testeren, og derfor kræver det ikke indblik i koden for at kunne black box teste.\\\\
White box test, også kendt som strukturel test, handler derimod om, hvorvidt systemet fra et indre perspektiv opfører sig som forventet.
Man vælger i white box test at have fokus på programmets logik og dette kan være med til at skabe en høj kodedækning, da man eksempelvis tester, om alle forgreninger kan udføres.
\\Til vores program har vi valgt at lave tre typer tests; unittest, kodedækningstest og performancetest. 
Det skal hertil nævnes, at der i projektet er blevet udført integrationstest, da vi ikke bare har valgt en \textit{big bang-approach} og først testet systemet til sidst.
\section{Unit test}
Unit test går i sin helhed ud på at teste de enkelte metoder i spillet, det er altså derfor en white box test. \\
Vores unit tests er dokumenteret i koden, således man kan se opbygningen på hvert enkelte test.
Nedenstående uddrag af vores kode beskriver en unit test af getAmount() og startAmount():\\
\begin{lstlisting}
    @Test
    public void getAmount() throws Exception {
        Stash stash1 = new Stash();
        Stash stash2 = new Stash(100);

        Assert.assertEquals(0, stash1.getAmount());
        Assert.assertEquals(100, stash2.getAmount());
    }
\end{lstlisting} 
\\ \\
Først oprettes stash1 af klassen Stash, og da den ikke initaliseres med en værdi, så forventer vi, at getAmount() returnerer en værdi på 0.
Ligeledes når stash2 bliver oprettet, så bliver den intialiseret med 100, og derfor forventes der også en returværdi på 100, når getAmount() kaldes.

\section{Coverage test}
Vi testede spillet til at have en kodedækning på <INDSÆT TEXT><INDSÆT TEXT><INDSÆT TEXT><INDSÆT TEXT><INDSÆT TEXT><INDSÆT TEXT><INDSÆT TEXT><INDSÆT TEXT>, hvilket er godt til at, at vi kan acceptere det.
Det er vigtigt at have et program med så høj kodedækning som muligt for at undgå død kode og evt. dårligt optimeret program.
<INDSÆT TEXT><INDSÆT TEXT><INDSÆT TEXT>

\section{Performancetest}
Når man skriver et program er det vigtigt at have in mente, hvilket system der skal kunne køre programmet, altså det er specielt vigtigt at skrive et program med høj performance, hvis det skal bruges på et ressourcesvagt system.
Ligeledes har vi lavet en performancetest af 'Monopoly-junior' for at tjekke ressourceforbruget.
<INDSÆT TEXT><INDSÆT TEXT><INDSÆT TEXT><INDSÆT TEXT><INDSÆT TEXT><INDSÆT TEXT><INDSÆT TEXT><INDSÆT TEXT>


\section{Testcases}
<INDSÆT TEXT><INDSÆT TEXT><INDSÆT TEXT><INDSÆT TEXT><INDSÆT TEXT><INDSÆT TEXT><INDSÆT TEXT><INDSÆT TEXT>