\subsection{FURPS+ \& MoSCoW}

\subsubsection{FURPS+}

FURPS+ er en metode til at kategorisere / klassificere krav. \\
Vi har i denne opgave brugt FURPS+ metoden til at kategorisere de krav, som er udformet fra opgavebeskrivelsen.
Vi har ikke valgt at medtage alle FURPS+ punkterne, men medbragt dem der giver mening for opgaven. \\
Herunder ses et eksempel på FURPS+, og hvad de enkelte kategorier kan indeholde.\footnote{FURPS+ tabellen der vises, er hentet fra tidligere CDIO2-rapport af samme gruppe.}

\begin{figure*}[ht]{
    \centering
\begin{tabular}{|c | p{0,05cm} p{2,8cm} |p{10,5cm}|}
       \hline
       \textbf{F}   &   Functionality   &&
       Egenskaber, ydeevne, sikkerhed                   \\
       \hline

       \textbf{U}   &   Usability       &&
       Menneskelige faktorer, hjælp, dokumentation      \\
       \hline

       \textbf{R}   &   Reliability     &&
       Fejlfrekvens, fejlretning, forudsigelighed       \\
       \hline

       \textbf{P}   &   Performance     &&
       Svartider, nøjagtighed, ydeevne, ressourceforbrug                                                      \\
       \hline

       \textbf{S}   &   Supportability  &&
       Anvendelighed, tilpasningsevne, vedligeholdbarhed                                                      \\
       \hline

       \textbf{+}   &                   &&              \\

       &&   Implementation: &   Ressourcebegrænsninger, sprog og værktøjer, hardware                     \\

       &&   Interface:      &   Begrænsninger forårsaget af kommunikation med eksterne systemer              \\

       &&   Operations:     &   Systemstyring i dets operationelle ramme                              \\

       &&   Packaging:      &   F.eks. en fysisk boks   \\

       &&    Legal:         &   F.eks. licenser       \\
       \hline
\end{tabular}}
\end{figure*}

\noindent Vi her herunder en oversigt over krav til opgaven, kategoriseret ved hjælp af FURPS+.
\\\\\textbf{Functionality:}\\
1. Spillet skal være en viderebygning på det tidligere udviklet spil fra CDIO 2.
6. Spillerne skal kunne lande på et felt, og fortsætte derfra på næste slag.
4. Der skal oprettes forskellige typer felter, samt en spilleplade.
5. Hvert felt skal påvirke spillernes pengebeholdning forskelligt, og have en udskrift om hvilket felt han ramte.
9. Der skal implementeres med regler fra et rigtigt Monopoly Junior spil.
\\\\\textbf{Usability:}\\
2. Man skal kunne spille 2-4 spillere.
\\\\\textbf{(+) - Implementation:}\\
8. Spillet skal kunne køre på DTU's databar-computere.
\\\\\textbf{(+) - Interface:}\\
7. Spillerne skal gå i ring rundt på felterne, på spillepladen.
5. Hvert felt skal påvirke spillernes pengebeholdning forskelligt, og have en udskrift om hvilket felt han ramte.

\pagebreak

\subsubsection{MoSCoW}

MoSCoW, er et værktøj som kan bruges til at prioritere krav.
Herunder ses et eksempel på MoSCoW, og hvad de enkelte kategorier kan indeholde.\footnote{MoSCoW tabellen der vises, er hentet fra tidligere CDIO2-rapport af samme gruppe.} \\\\

\begin{tabular}{lll}
    \textbf{Mo} &   
    "Must have"                 &
    De mest vitale krav, vi ikke kan undgå. \\

    \textbf{S}  &   
    "Should have"               & 
    Vigtige krav, som ikke er vitale. \\

    \textbf{Co} &   
    "Could have"                & 
    The 'nice-to-haves' \\

    \textbf{W}  &   
    "Won’t have (this time)"    & 
    Things that provide little to no value you can give up on \\

\end{tabular}
\\\\

\noindent Vi her herunder en oversigt over krav til opgaven, prioriteret ved hjælp af MoSCoW.

\begin{center}
    \begin{tabular}{ | l | p{13cm} |}
    \hline
    \textbf{Must have}
    &        
        2. Man skal kunne spille 2-4 spillere.

        4. Der skal oprettes forskellige typer felter, samt en spilleplade.
        
        6. Spillerne skal kunne lande på et felt, og fortsætte derfra på næste slag.
        
        8. Spillet skal kunne køre på DTU’s databar-computere.
    \\
    
    \hline
    \textbf{Should have}
    &
        1. Spillet skal være en viderebygning på det tidligere udviklet spil fra CDIO 2.

        3. Der skal implementeres væsentlige elementer fra Monopoly Junior.

        5. Hvert felt skal påvirke spillernes pengebeholdning forskelligt, og have en ud-skrift om hvilket felt han ramte.

        7. Spillerne skal gå i ring rundt på felterne, på spillepladen.
    \\
    \hline
    \textbf{Could have}
    &
        9. Der skal implementeres med regler fra et rigtigt Monopoly Junior spil.
    \\
    \hline
    \textbf{Won't have}
    &

    \\
    \hline
    \end{tabular}
\end{center}

\noindent Vi har ikke nogen krav der kan sættes i kategorien won't have.
Dette felt bruges ofte til krav der bliver stillet fra kunden, som man ikke kan udføre, eller har svært ved pga. ressourcer eller lign.
Der er selvfølgelig plads til videreudvikling i vores, hvilket ville ligge under won't have. 