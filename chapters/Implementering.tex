\chapter{Implementering}
Vi har i dette projekt taget udgangspunkt i CDIO2, som er det første projekt, hvor vi bl.a. virkelig har gjort brug af planlægningsværktøjet Asana og kommunikationsværktøjet Slack.
I det her projekt var det nødvendigt for os at programmere, sideløbende med en analyse- og designfase.
Det viste sig at være en kæmpe hjælp at have et klassediagram, mens man koder, fordi man på klassediagrammet har en idé om, hvordan koden skal se ud.
\\
Det er dog klart, at diagrammerne bliver ændret i løbet af kodningen, fordi det er irrationelt at tro, at uforudsete ting ikke kan forekomme.
I 'Udviklingsmetoder til IT-Systemer' har vi altså lært ikke at følge en vandfaldsmodel, men en agil model, hvor projektet deles op i iterationer, som det vises i kapitlet \textit{Projektplanlægning.}.
\\
Vi har prøvet til vores bedste evner at følge GRASP-modellen, således vi får en en attraktiv low coupling og high cohesion.

\section{Brugervejledning for programmet}
Programmets brugerinterface er meget simpelt sat op, og guider brugeren nemt igennem hvad han/hun skal gøre.
Når man starter programmet op, vil det åbne en GUI, hvor den vil starte med at spørge, hvor mange spillere, der skal spille spillet (spillet tager imod 2, 3 og 4 spillere).
Dette indtaster brugeren og trykker herefter på ok, hvorefter spillet spørger om spiller 1's navn og ønsker derefter at vide, hvilken \textit{type} spiller, han/hun er.
Det samme gør sig gældende for de næste spillere i samme rækkefølge.
\\
Spillet fortæller nu, at det er spiller 1's tur, hvorefter man skal trykke på knappen \textit{kast}, således at der kommer en visualisering af et terningekast og et tekstfelt med terningens værdi.
Herefter vil programmet opdatere spillerens score, hvis altså han/hun har købt en grund eller skal betale leje, og give turen videre.
Dette fortsætter programmet med, indtil der kun er én spiller med en positiv pengebeholdning.
Brugeren kan desuden holde musen over de forskellige felter, for at få beskrivelsen af feltet, og dens leje.