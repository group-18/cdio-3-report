\chapter{Implementering}
Vi har i dette projekt taget udgangspunkt i CDIO2, som er det første projekt, hvor vi bl.a. virkelig har gjort brug af planlægningsværktøjet Asana og kommunikationsværktøjet Slack.
I det her projekt var det nødvendigt for os at programmere, sideløbende med en analyse- og designfase.
Det viste sig at være en kæmpe hjælp at have et klassediagram, mens man koder, fordi man på klassediagrammet har en idé om, hvordan koden skal se ud.
\\
Det er dog klart, at diagrammerne bliver ændret i løbet af kodningen, fordi det er irrationelt at tro, at uforudsete ting ikke kan forekomme.
I 'Udviklingsmetoder til IT-Systemer' har vi altså lært ikke at følge en vandfaldsmodel, men en agil model, hvor projektet deles op i iterationer, som det vises i kapitlet \textit{Projektplanlægning.}.
\\
Vi har prøvet til vores bedste evner at følge GRASP-modellen, således vi får en en attraktiv low coupling og high cohesion.

\section{Brugervejledning for programmet}
Programmets brugerinterface er meget simpelt sat op, og guider brugeren nemt igennem hvad han/hun skal gøre.
Når man starter programmet op, vil det åbne en GUI, hvor den vil starte med at spørge, hvor mange spillere, der skal spille spillet (spillet tager imod 2, 3 og 4 spillere).
Dette indtaster brugeren og trykker herefter på ok, hvorefter spillet spørger om spiller 1's navn og ønsker derefter at vide, hvilken \textit{type} spiller, han/hun er.
Det samme gør sig gældende for de næste spillere i samme rækkefølge.
\\
Spillet fortæller nu, at det er spiller 1's tur, hvorefter man skal trykke på knappen \textit{kast}, således at der kommer en visualisering af et terningekast og et tekstfelt med terningens værdi.
Herefter vil programmet opdatere spillerens score, hvis altså han/hun har købt en grund eller skal betale leje, og give turen videre.
Dette fortsætter programmet med, indtil der kun er én spiller med en positiv pengebeholdning.
Brugeren kan desuden holde musen over de forskellige felter, for at få beskrivelsen af feltet, og dens leje.

\section{Verificering ift. krav og validering ift. kunde}
I dette afsnit vil der forekomme en mindre diskussion ift. verificering af at produktet lever op til de opstillede krav, samt validering af krav fundet i oplæget og hvorvidt de stemmer over ens med kundens ønsker.\\

\subsubsection{Verificering ift. krav}
Verificering i forhold til krav tager udgangspunkt i kravspecifikationen fundet under afsnittet analyse, og tager hånd om på de forskellige opstillede krav, og hvorvidt produktet lever op til disse krav. Udfra det dannes et helheds billede over opfyldte krav, og er der evt. mangler noteres disse.\\
Generelt så har vi bestræbet os efter at få opfyldt det største antal krav muligt. I alt er der tale om 25 krav, hvori de essentielle krav for at spillet følger reglerne blev prioritereret. På baggrund af mangel på tid samt større besværligheder har det slet ikke været muligt, at implementerer de mere avancerede krav, nemlig krav nr. 22, 23, 24 samt 25. Størstedelen af kravene der rækker fra 1-21 har vi prøvet så vidt som muligt at implementere, dog skal det tages i betragtning at der muligvis er et par krav der ikke blev implementeret. Jævnfør afsnittet 'Implementerede krav' under kravspecifikation.\\

\subsubsection{Validering ift. kunde}
Som udtrykt i spillereglerne fra Matador Junior, er der tale om andre regler end dem man ellers bruger i et normalt matador. Kunden ønsker et Matador Junior spil, hvor det forrige del bliver udbygget med forskellige typer af felter, samt en decideret spilleplade. Derudover skal spillerne kunne lande på et felt og så fortsætte derfra på næste slag. Spillerne går i ring på brættet, hvorpå antal spillere må ikke være mindre end 2 og større end 4.
\\
IOOuterActive mener at spillet lever op til kundens forventninger. Et matador spil der følger reglerne fra Matador Junior, samt lever op til kundens krav er blevet udviklet.\\