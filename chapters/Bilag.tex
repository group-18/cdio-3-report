\chapter{Bilag}

\section{Reglementet til Matador Junior}
Herunder ses de officielle regler til Matador Junior-spillet, som vi har baseret vores spil på:
\textbf{Sådan kommer I i gang med at spille Junior Matador:} \footnote{Spillerreglerne.dk, 2014, http://spillereglerne.dk/junior-matador-spilleregler/}
\\
\begin{enumerate}
    \item Hver spiller vælger en bil
    \item Vælg en af spillerne til at være bankdirektør. 
    Bankdirektøren giver hver spiller 12.000 kr. som startpenge. 
    Banken beholder resten af pengene og eventyrkortene. 
    Bankdirektøren foretager alle ind- og udbetalinger i spillet med undtagelse af lejeindtægterne, som naturligvis betales direkte til ejerne af eventyrkortene.
    \item Læg prøv lykken-kortene i en bunke på spillepladen med bagsiden opad.\\
\end{enumerate}
Stil den hvide bil på startfeltet på inderbanen. 
Bilen bruges til at markere spilletiden.
Hver gang man passerer START, skal man flytte den hvide bil et felt fremad på inderbanen. 
Den ældste deltager har til opgave at holde øje med, at bilen flyttes et felt frem, hver gang en spiller passerer START.\\

\textbf{Selve spillet}
\\Deltagerne stiller deres bil på feltet START og bliver enige om, hvem der skal begynde. 
Spillet fortsætter derefter i urets retning.\\

\textbf{En spilleomgang}
\\Spilleren kaster begge terninger og flytter sin bil lige så mange felter frem, som terningernes øjne viser. 
Når man har fulgt anvisningerne for det felt, som man landede på, går turen videre til næste deltager.\\

\textbf{Køb af eventyrkort}
\\Når man lander på et eventyrfelt, som ikke ejes af nogen anden deltager, kan man købe dette af banken for den pris, der står på feltet. 
Man får eventyrkortet af banken og lægger det med billedsiden opad foran sig på bordet. 
I overensstemmelse med de beløb, der er angivet på eventyrkortet, kan man nu opkræve leje af de spillere, der lander på feltet. 
Kan eller vil en spiller ikke købe eventyrkortet, sker der ingenting, og turen går videre til næste deltager.\\

\textbf{Sådan betaler du leje}
\\Hver gang man lander på et felt, der ejes af en anden deltager, skal der betales leje. Beløbet fremgår af eventyrkortet. Glemmer man at opkræve leje af en medspiller, har man tabt retten til at opkræve lejen, så snart den næste spiller har kastet terningerne. Ejer en spiller alle tre eventyrkort i samme serie, bliver lejen højere (se beløbene på eventyrkortene).\\

\textbf{Prøv lykken}
\\Lander man på et felt med PRØV LYKKEN, skal man tage det øverste kort i bunken og følge anvisningerne på kortet. 
Læg kortet tilbage nederst i bunken, når det er brugt.\\

\textbf{Feltet med fængsel}
\\Lander man på dette felt, sker der ingenting. 
Man er bare på besøg og flytter som sædvanligt næste gang, det er ens tur. Feltet med “byt kort”
Lander du på dette felt, må du bytte et eventyrkort med en spiller efter eget valg. 
Tag et af dine egne eventyrkort og byt det med et af den anden spillers kort. 
Det er dig der bestemmer, hvilket kort du vil bytte det til, og den anden spiller kan ikke sige nej til dette kortbytte. 
Er du heldig, kan du bytte så du får en komplet serie. 
Når en spiller har alle tre kort i en serie, er kortene beskyttede.\\

\textbf{Feltet med Taxa}
\\Lander en spiller på et felt med en taxa, må spilleren straks køre med taxa til et eventyrfelt på spillepladen efter eget valg. 
Køreturen koster 1.000 kr., som spilleren betaler til banken. 
Du skal følge anvisningerne for det felt du “kører” hen til, inden du afslutter din tur. 
Kører du f.eks. til et eventyrfelt, som ikke ejes af en anden deltager må du købe feltet.\\

\textbf{Startfeltet}
\\Hver gang en spiller passerer START, får han/hun 3.000 kr. af banken. Man skal huske at flytte den hvide bil et felt fremad på inderbanen, hver gang en spiller passerer start. Alle kort i en serie
Har en spiller alle 3 kort i samme serie, kan spilleren opkræve højere leje. Desuden er serien beskyttet, dvs. at ingen anden spiller kan tage eller bytte kort fra denne serie.\\

\textbf{Ekstrakast}
\\Slår en spiller 2 ens med terningerne (f.eks. 2 seksere), får spilleren et ekstrakast. 
Før spilleren kaster terningerne igen, skal han/hun først følge anvisningerne, der er angivet på det felt, som spilleren først landede på. 
Efter ekstrakastet skal spilleren så følge de anvisninger, der er angivet på det nye felt, som spilleren er flyttet hen på. 
Kaster man 3 gange i træk 2 af samme slags, må man ikke flytte tredje gang, men skal gå direkte i fængsel og springes en omgang over. 
Turen går derefter videre til næste deltager.\\

\textbf{Pengemangel}
\\Hvis en spiller skylder en anden deltager flere penge, end han/hun ejer, er spilleren nødt til at sælge et eller flere af sine eventyrkort tilbage til banken. 
Salgsprisen for et eventyrkort er altid den samme som indkøbsprisen. 
Salg af eventyrkort er ikke tilladt indbyrdes mellem spillerne.\\

\textbf{Hvem vinder?}
\\Når den hvide bil kører i mål, er spillet slut. 
Spillerne tæller nu værdien sammen af deres kontanter samt de eventyrkort de ejer (indkøbsprisen). 
Den rigeste spiller vinder, og bliver dermed spillets Junior Matador.