\renewcommand{\abstractname}{Abstract}

\begin{abstract}
\noindent As with both CDIO1 and CDIO2, this project is combining the three IT-related courses \textit{Development methods for IT-systems}, \textit{Introductory programming}, and \textit{Version control and test methods}.
Like the former project the purpose is to create the well-known 'Monopoly'-game, and whilst the latter projects were parts of a game, this project combines those parts into 'Monopoly-junior'.
\\
We succeeded creating the 'Monopoly-junior'-game without sacrificing any requirements, thus fulfilling the requirements of the customer and the integrity of a real 'Monopoly'-game which is known for its key components such as choosing the type of player you are and letting the youngest player start.
\\
The fundamental methods such as throwing a die, changing the players' names et cetera have all been unit tested and been documented in the code with a corresponding javadoc.
\\
The game has been analyzed and designed with the help of usecase-, sequence-, and classdiagrams all written in the UML-notation as part of a agile development proces to simulate the industry standard.
\\
To conclude, we are all pleased with the different phases of the development process as well as the final product.
    \ThisLRCornerWallPaper{0.7}{graphics/dtu/DTU-frise-SH-15}
\end{abstract}