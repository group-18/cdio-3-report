\chapter{Konfigurationsstyring}

I dette afsnit berører vi konfigurationen af programmet.
En beskrivelse af de forskellige krav, samt nødvændige programmer for at køre systemet. 
\\Dette inkluderer både kørsel af spillet, men også vedligeholdelse af programmet, med compiling, installation.
Vi kommer samtidigt ind på hvorledes koden importeres fra et git repository.
Konfigurationsstyring er delt op i to forskellige platforme. \\

\subsubsection{Udviklingsplatformen}

I vores udviklingsplatform har vi brugt følgende software:

\noindent - Apple MacOS High Sierra \& Microsoft Windows 10 \\
- IntelliJ - Version 2017.2.6 \\
- JAVA (jre-8u151 \& jdk-9.0.1) \\
- matadorgui.jar (Hentet via Maven)\footnote{\url{https://github.com/diplomit-dtu/Matador_GUI/raw/repository}}\\

\subsubsection{Produktionsplatformen}
Produktionsplatformen er alt det software der skal bruges til at køre vores færdige program:

\noindent - Operativsystem der kan køre JAVA\\
- JAVA (jre-8u151 \& jdk-9.0.1) \\

\noindent Skal man så videreudvikle på programmet skal man brug andre konfigurationer:\\
\noindent - Apple MacOS High Sierra \& Microsoft Windows 10 \\
- IntelliJ - Version 2017.2.6 (kan være anden IDE, dog er der medfølgende indstillinger i IntelliJ, som der er fordele i. \\
- JAVA (jre-8u151 \& jdk-9.0.1) \\
- matadorgui.jar (Hentet via Maven)\footnote{\url{https://github.com/diplomit-dtu/Matador_GUI/raw/repository}}\\


% \subsubsection{Minimumskrav}

% Udviklingsplatformen har vi brugt intelliJ  IDEA er alt det software vi bruger under udviklingen af vores projekt.
% Herunder ses et minimumskrav til kørsel af JAVA 8, som programmet er programmeret i. Programmet køres også igennem en .jar fil\\

% \textbf{Windows}
% \\Windows Vista SP2, 7 SP1, 8.x (Desktop), 10 (8u51 and above)
% \\Windows Server Server 2012 \& 2012 RS2 (64bit)
% \\RAM: 128 MB, 
% \\Disk space: 124 MB for JRE; 2 MB for Java Update
% \\Processor: Minimum Pentium 2 266 MHz processor
% \\Browsers: Internet Explorer 9 and above, Firefox \\

% \textbf{Mac OS X}
% \\Intel-based Mac running Mac OS X 10.8.3+, 10.9+
% \\Administrator privileges for installation
% \\64-bit browser (Safari, for example) is required to run Oracle Java on Mac. \\

% \textbf{Linux}
% \\Oracle Linux 5.5+1, 6.x (32-/64-bit), 7.x(64-bit) (8u20 and above)
% \\Red Hat Enterprise Linux 5.5+1, 6.x (32-bit), 6.x (64-bit)2
% \\Red Hat Enterprise Linux 7.x (64-bit)2 (8u20 and above)
% \\Suse Linux Enterprise Server 10 SP2+, 11.x, 12.x (64-bit)2 (8u31 and above)
% \\Ubuntu Linux 12.04 LTS, 13.x, 14.x (8u25 and above)
% \\Ubuntu Linux 15.04 (8u45 and above), 15.10 (8u65 and above)
% \\Browsers: Firefox \\

\newpage
\section{Konfigurationsvejledning}
Følgende procedure er målrettet et Eclipse miljø, taget i betragtning at projektet skal kunne iscenesættes i Eclipse. Samtidigt skal projektet importeres fra et GIT repository, hvor så en vejledning for dette følger i dette afsnit.\footnote{ Da CDIO3 er blot en revideret udgave af CDIO2, er der tale om en magen konfigurationsvejledning til den fundet i CDIO2, og som resultat, er følgende vejledning baseret på vejledningen fra CDIO2 rapporten.}\\

\noindent \textbf{Hente projekt fra repository, Git}\\
Det er flere måder at hente et projekt fra et Git repository, hvor det i sidste ende afhænger af om man gør brug af en lokal installation eller web udgave af det omtalte Git klient. I vores tilfælde havde vi gjort brug af Github til håndtering af vores Git repository. Ved lokalisering af projektet på Github skal man blot klikke på ‘Clone or download’, herefter hentes en lokal kopi af det omtalte projekt. For at kunne åbne projektet i eksempelvis Eclipse, skal projektet blot importeres.\\

\noindent \textbf{Compiling, installering og afvikling af kildekoden, Eclipse}\\
Eftersom at projektet er tilgængelig på maskinen lokalt, skal man blot importerer projektet i Eclipse. Dette gøres ved at åbne Eclipse og lokaliserer menuen ‘File’ på menubjælken. Under ‘File’ klikkes derefter på ‘Import’, og typen ‘Projects from Folder or Archive’ vælges. Mappen der indeholder vores lokale Git kopi lokaliseres og sættes som kilde, og processen afsluttes vha. knappen ‘Finish’.\\

\noindent \textbf{\textit{Processen ser som følger:}}\\
\textit{File -> Import -> Projects from Folder or Archive -> Source: Git kopi -> Finish}\\

\noindent For at kunne compile og afvikle programmet, skal man blot klikke på det ikon, der repræsenterer en trekant omringet af en grøn cirkel. Eventuelt kan man nøjes med at klikke lokaliserer menuen ‘Run’ på menubjælken.
\newpage
\section{Vejlednig til eksport samt kørsel af program}
Følgende er en vejledning til hvordan man compiler en såkaldt .jar file (java archive) ved brug af Eclipse. Resultatet vil da være en java fil der kan afvikles via eksempelvis terminal. Selve kildekoden pakkes ind i en .jar fil.\\

\noindent Primært er der to måder at compile en runnable jar fil, enten ved brug af terminal/commando prompt eller via eksport funktionen i Eclipse. Vi vælger at tage udgangspunkt i det sidst nævnte, og vejledningen ser som følger:\\

\subsection{Eksport af program}
\begin{enumerate}   
    \item Åbn projekt i Eclipse og lokaliserer menuen ‘File’ på menubjælken.
    \item Vælg derefter eksport funktionen under menuen ‘File’.
    \item Find ‘Java’ eksport konfigurationen, vælg derefter ‘Runnable JAR file’.
    \item I det ny opstået vindue vælges opstarts konfigurationen, hvori der specificeres hvilken ’Java Application’ der skal opfattes som start.
    \item I samme vindue vælges også sti for eksport destination. Derudover skal der også tages et valg i forhold til hvordan biblioteket skal behandles. 
    \item Ved tryk på ‘Finish’ eksporter Eclipse med udgangspunkt i den valgte konfiguration projektet til en runnable jar fil.\\
\end{enumerate}

\subsection{Kørsel af program}
\begin{enumerate}   
    \item For at åbne/afvikle den tidligere eksporteret jar fil skal man blot køre programmet i terminal/commando prompt.
    \item I terminalen skrives følgende: \textit{java -jar ‘navn på jar-fil‘.jar}.
    \item Herefter burde programmet hvis alt går vel starte op.\\    
\end{enumerate}